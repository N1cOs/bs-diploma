\usepackage{fontspec} % XeTeX
\usepackage{xunicode} % Unicode для XeTeX
\usepackage{pdfpages} % Вставка PDF

% Шрифт
\defaultfontfeatures{Ligatures=TeX}
\setmainfont{Times New Roman}
\newfontfamily\cyrillicfont{Times New Roman}

% Русский язык
\usepackage{polyglossia}
\setdefaultlanguage{russian}

\renewcommand{\baselinestretch}{1.5} % Межстрочный интервал
\parindent 1.25cm % Абзацный отступ
\clubpenalty=10000 % Запрещаем разрыв страницы после первой строки абзаца
\widowpenalty=10000 % Запрещаем разрыв страницы после последней строки абзаца
\sloppy % Запрещаем выходить за границы страницы
\hyphenpenalty=1000 % Частота переносов

% Отступы у страниц
\usepackage{geometry}
\geometry{left=3cm}
\geometry{right=1.5cm}
\geometry{top=2cm}
\geometry{bottom=2cm}

% Подсчет количества страниц
\usepackage{lastpage}
\setcounter{page}{7}

% Сбрасываем счетчик таблиц и рисунков в каждой новой главе
\counterwithin{figure}{section}
\counterwithin{table}{section}

% Оформление оглавления
\usepackage{tocloft}
\renewcommand\cftsecleader{\cftdotfill{\cftdotsep}} % Точки для секций
\renewcommand\cfttoctitlefont{\normalsize\bfseries\hspace{0.38\textwidth}\MakeTextUppercase} % Выравниваем название по центру и приводим в верхний регистр 
\renewcommand\cftsecfont{\mdseries} % Названия разделов не жирные
\renewcommand\cftsecpagefont{\mdseries} % Номера страниц не жирные
\setcounter{tocdepth}{3} % Глубина оглавления до subsubsection

% Заголовки секций в оглавлении в верхнем регистре
\usepackage{textcase}
\makeatletter
\let\oldcontentsline\contentsline
\def\contentsline#1#2{
    \expandafter\ifx\csname l@#1\endcsname\l@section
        \expandafter\@firstoftwo
    \else
        \expandafter\@secondoftwo
    \fi
    {\oldcontentsline{#1}{\MakeTextUppercase{#2}}}
    {\oldcontentsline{#1}{#2}}
}
\makeatother

% Оформление заголовков
\usepackage[compact,explicit]{titlesec}
\titleformat{\section}{\normalsize\bfseries}{}{1.25cm}{\thesection\hspace{.5em}\MakeTextUppercase{#1}}
\titleformat{\subsection}[block]{\normalsize\bfseries}{}{1.25cm}{\thesubsection\hspace{.5em}#1}
\titleformat{\subsubsection}[block]{}{}{1.25cm}{\thesubsubsection\hspace{.5em}#1}
\titleformat{\paragraph}[block]{\normalsize\bfseries}{}{0em}{\MakeTextUppercase{#1}}

\titlespacing{\subsection}{0pt}{.5em}{.25em}
\titlespacing{\subsubsection}{0pt}{.5em}{0em}

% Изображения
\usepackage{graphicx}
\graphicspath{{images/}} % Путь до директории с изображениями
\renewcommand{\thefigure}{\thesection.\arabic{figure}} % Формат именования рисунков: секция.номер
\addto\captionsrussian{\renewcommand{\figurename}{Рисунок}}

\usepackage{caption}
\DeclareCaptionLabelSeparator{hyphen}{ -- }
\captionsetup[figure]{justification=centering, labelsep=hyphen, format=plain} 

% Списки
\usepackage{enumerate}
\usepackage{enumitem}
\setlist{nolistsep} % Убираем лишний отступ между соседними пунктами списка
\setlist[enumerate]{label=\arabic*)} % Устанавливаем формат пункта в списке '1)'

% Библиография
\usepackage[numbers,colon]{natbib}
\bibliographystyle{styles/utf8gost705u}
\makeatletter
\renewcommand{\@biblabel}[1]{#1.} % Меняем стиль нумерации '[1]' -> '1.'
\renewcommand{\bibsection}{} % Удаляем наименование по умолчанию
\makeatother

% Оформление исходного кода
\usepackage{listings}
\setmonofont{FreeMono} % Шрифт для листингов
\lstset{
    basicstyle=\small\ttfamily,
    breaklines=true,
    tabsize=2,
    frame=single,
    numbers=left,
}
\DeclareCaptionFormat{listing}{
\parbox{\textwidth}{#1#2#3}
}
\captionsetup[lstlisting]{
    format=listing,
    skip=7pt
}

% Секции без номеров (введение, заключение...)
\newcommand{\anonsection}[1]{
    \phantomsection % Корректный переход по ссылкам в содержании
    \paragraph{\centerline{{#1}}}
    \addcontentsline{toc}{section}{#1}
}

% Добавление рисунка 
\newcommand{\addimg}[4]{ 
    \begin{figure}[h!]
        \centering
        \includegraphics[width=#2\linewidth]{#1}
        \caption{#3} \label{#4}
    \end{figure}
}
