\anonsection{Введение}

С каждым днем мы все чаще сталкиваемся с устройствами Интернета вещей (Internet of Things, IoT), которые призваны упростить 
нашу повседневную жизнь. Концепция IoT подразумевает подключение различных устройств между собой, которые обмениваются 
данными как в рамках внутренней сети, так и внешней. Классическим примером данной концепции является "умный" дом, когда, 
например, мы можем регулировать уровень освещенности в комнате с помощью голосового управления на телефоне. 

Однако технологии Интернета вещей внедряются не только в нашу обыденную жизнь, но и в промышленности. Так называемый 
промышленный Интернет вещей (Industrial Internet of Things, IIoT) уже сегодня позволяет повысить эффективность процессов 
производства. Типичный производственный объект может быть источником большого количества данных, собранных с датчиков, 
анализ которых позволит обнаружить и устранить проблему на ранних этапах ее появления. В качестве примера внедрения IIoT 
можно вспомнить компанию Amazon, которая активно применят роботов и различные датчики на своих складах для более быстрого 
поиска нужных товаров \cite{AMAZON_WH}.

На сегодняшний день устройства IoT и IIoT чаще всего применяются в связке с облачной инфраструктурой. То есть данные, 
собираемые устройствами отправляются в "облако", где дальше уже сам анализ происходит на машинах с большими вычислительными
ресурсами. Такой подход больше применим для долговременной аналитики, но если результат обработки нужен "здесь и сейчас", 
то задержки при передачи данных от устройств до центров обработки данных (ЦОД) могут быть критичны. В связи с этим начали 
набирать популярность граничные вычисления (edge computing), когда обработка данных происходит рядом с конечными устройствами. 
Так как количество данных с каждым годом будет только расти, то можно утверждать, что граничные вычисления будут находить 
все более широкое применение. Согласно исследованиям компании Gather, к 2025 году около 75\% данных будет обрабатываться 
на границе, не доходя до ЦОД \cite{GATHER_RCH}.

Часто для анализа данных собранных с устройств IoT и IIoT применяются методы глубокого обучения, которые позволяют 
установить более сложные зависимости во входных данных. Глубокое обучение подразумевает построение искусственных нейронных 
сетей для решения прикладных задач. Методы глубокого обучения появились еще в 80-х годах, но начали стремительно набирать
популярность около 2010 года. Обусловлено это несколькими причинами. Во-первых, с ростом Интернета увеличивалось количество
данных, которые можно было использовать для обучения. Во-вторых, с развитием графических процессоров (Graphics Processing Unit, GPU)
процесс обучения нейронных сетей сократился в десятки раз. Впоследствии предобученные модели могут запускаться на устройствах
и без GPU, но при этом демонстрировать приемлемое качество работы.

Исторически задача распознавания объектов на изображениях являлась тяжелой, но с развитием методов глубокого обучения удалось 
повысить эффективность решения данной задачи. Сегодня предобученные модели для распознавания объектов могут быть запущены даже 
на компьютерах с ограниченными вычислительными ресурсами (КОВР). В данной работе под КОВР подразумевается одноплатный компьютер 
без каких-либо специальных модулей для аппаратного ускорения методов машинного обучения. Классическим примером КОВР является одноплатный
компьютер Raspberry Pi. Чаще всего КОВР собирают данные с различных устройств, проводят их первичную предобработку и далее отправляют
их для полного анализа в "облако". Однако при текущем развитии КОВР, их вычислительные ресурсы могут использоваться для куда более
сложных вещей, таких как распознавания объектов на видеоизображениях. Вполне ожидаемо, что одному экземпляру КОВР решить 
такую задачу будет сложно, но при объединении нескольких экземпляров КОВР в кластер можно добиться приемлемых результатов. 
Основными преимуществами использования кластера из КОВР являются:

\begin{enumerate}
\item снижение затрат на анализ видеоизображений. Во-первых, если на объекте уже имеются экземпляры КОВР, 
то для решения этой задачи может использоваться существующая инфраструктура. Во-вторых, развертывание кластера из КОВР
требует меньше финансовых ресурсов по сравнению с долговременным использованием облачной инфраструктуры; 
\item уменьшение задержек при передаче видеоизображений, что способствует увеличению отзывчивости систем, которые 
опираются на анализ этих данных; 
\item повышение конфиденциальности собранных данных за счет их локальной обработки.
\end{enumerate}

Приведенные выше утверждения говорят об \textbf{актуальности} данной работы. Главной \textbf{проблемой} на сегодняшний день 
является отсутствие в открытом доступе каких-либо полноценных решений для задачи распределенного распознавания объектов на 
видеоизображениях на кластере из КОВР. В связи с этим, \textbf{целью} данной работы является организация распределенного 
распознавания объектов на видеоизображениях на кластере из КОВР. Для достижения поставленной цели требуется решить 
следующие \textbf{задачи}:

\begin{enumerate}
\item провести обзор методов для распознавания объектов на изображениях;
\item провести анализ особенностей архитектуры КОВР;
\item разработать алгоритм для распределения вычислений, необходимых для распознавания объектов на видеоизображениях, 
между узлами кластера;
\item провести тестирование разработанного алгоритма на кластере из КОВР.
\end{enumerate}

\clearpage
